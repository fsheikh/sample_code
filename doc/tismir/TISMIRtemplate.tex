%%%%%%%%%%%%%%%%%%%%%%%%%%%%%%%%%%%%%%%%%%%%%%%%%%%%%%%%%%%%%%%%%%%%%%%%%%%%%%%%
% Template for TISMIR Papers
% 2017 version, based on previous ISMIR conference template
%%%%%%%%%%%%%%%%%%%%%%%%%%%%%%%%%%%%%%%%%%%%%%%%%%%%%%%%%%%%%%%%%%%%%%%%%%%%%%%%

\documentclass{article}

%%%%%%%%%%%%%%%%%%%%%%%%%%%%%%%%%%%%%%%%%%%%%%%%%%%%%%%%%%%%%%%%%%%%%%%%%%%%%%%%
% Sample Document LaTeX packages
%%%%%%%%%%%%%%%%%%%%%%%%%%%%%%%%%%%%%%%%%%%%%%%%%%%%%%%%%%%%%%%%%%%%%%%%%%%%%%%%

\usepackage[utf8]{inputenc}
\usepackage{tismir}
\usepackage{amsmath}
\usepackage{hyperref}
\usepackage{url}
\usepackage{graphicx}
\usepackage{booktabs}
\usepackage{lipsum}
\usepackage{forest}

%%%%%%%%%%%%%%%%%%%%%%%%%%%%%%%%%%%%%%%%%%%%%%%%%%%%%%%%%%%%%%%%%%%%%%%%%%%%%%%%
% Title and Author information
%%%%%%%%%%%%%%%%%%%%%%%%%%%%%%%%%%%%%%%%%%%%%%%%%%%%%%%%%%%%%%%%%%%%%%%%%%%%%%%%

\title{QawwalRang: A dataset for genre classification of Qawwali}
%
\author{%
Anonymous }%

\date{}


%%%%%%%%%%%%%%%%%%%%%%%%%%%%%%%%%%%%%%%%%%%%%%%%%%%%%%%%%%%%%%%%%%%%%%%%%%%%%%%%
% Additional Paper Information
%%%%%%%%%%%%%%%%%%%%%%%%%%%%%%%%%%%%%%%%%%%%%%%%%%%%%%%%%%%%%%%%%%%%%%%%%%%%%%%%

% Article Type - Uncomment and modify, if necessary.
% Accepted values: research, overview, and dataset
\type{dataset}

% Citation in First Page
%
% "Mandatory" (if missing will print the complete list of authors,
% including the \thanks symbols)
\authorref{Anonymous,~A.}
%
% (Optional)
% \journalyear{2017}
% \journalvolume{V}
% \journalissue{N}
% \journalpages{xx--xx}
% \doi{xx.xxxx/xxxx.xx}

% Remaining Pages (Optional)
%
\authorshort{Anonymous, ~A.} %or, e.g., \authorshort{Author1 et al}
% \titleshort{Template for TISMIR}


%%%%%%%%%%%%%%%%%%%%%%%%%%%%%%%%%%%%%%%%%%%%%%%%%%%%%%%%%%%%%%%%%%%%%%%%%%%%%%%%
% Document Content
%%%%%%%%%%%%%%%%%%%%%%%%%%%%%%%%%%%%%%%%%%%%%%%%%%%%%%%%%%%%%%%%%%%%%%%%%%%%%%%%

\begin{document}

%%%%%%%%%%%%%%%%%%%%%%%%%%%%%%%%%%%%%%%%%%%%%%%%%%%%%%%%%%%%%%%%%%%%%%%%%%%%%%%%
% Abstract
%%%%%%%%%%%%%%%%%%%%%%%%%%%%%%%%%%%%%%%%%%%%%%%%%%%%%%%%%%%%%%%%%%%%%%%%%%%%%%%%

\twocolumn[{%
%
\maketitleblock
%
\today
%

\begin{abstract}
A new dataset containing 72 Qawwali songs of one-minute duration along with tagged metadata is presented. Majority of samples in the dataset exhibit strong similarity in the fundamental musical properties. This is demonstrated by using the dataset for Qawwali genre detection against popular western music genres. Qawwali detection is a heuristic based unsupervised genre classification algorithm based on source separation of two major components typical of Qawwali performances namely Tabla and Taali. The algorithm shows an \textit{Accuracy} mean across genres of almost 90\% with a \textit{Recall} of 76\%. Dataset release is accompanied with python programs used to construct and extend the dataset as well as reproducing the results presented in this study. \end{abstract}
%
\begin{keywords}
qawwali, musical dataset, music genre classification
\end{keywords}
}]
\saythanks{}

%%%%%%%%%%%%%%%%%%%%%%%%%%%%%%%%%%%%%%%%%%%%%%%%%%%%%%%%%%%%%%%%%%%%%%%%%%%%%%%%
% Main Content Start
%%%%%%%%%%%%%%%%%%%%%%%%%%%%%%%%%%%%%%%%%%%%%%%%%%%%%%%%%%%%%%%%%%%%%%%%%%%%%%%%

\section{Introduction}\label{sec:intro}

Qawwalis have a long and historic tradition in Indian subcontinent music, by some accounts they go back to at least 13$^{th}$ century. One of the earliest Indian classical music families "Qawal Baccha Ghrana" \citep{JayashreeBhat} is credited with refining this singing form as a distinctive genre. Qawwalis been used to recite devotional songs, they have been sung by Qawwal groups on shrines of religious figures across South Asia. Even today they are popular in Indian subcontinent with performances given on happy as well as sad occasions. They are part and parcel of pop culture both in India and Pakistan with songs of this genre often included as part of film and commercial music.

Qawwali performances can incorporate both classical and modern musical elements, however the raga oriented melodies employed herein is similar to many other classical genres like Khayal, Ghazal or Thumri \citep{qureshi1986sufi}. Two perceptually differentiating factors of Qawwali performance are both related to rhythm and beat patterns. Similar to other Indian classical music genres, the percussion instrument Tabla \endnote{Tabla \\  \url{https://sriveenavani.com/Tabla}} supplies rhythm, but in Qawwali genre the rhythm patterns (Taals) are quite limited in scope. E.g. most Qawwalis employ Keherwa \endnote{Keherwa \\ \url{https://www.taalgyan.com/taals/keherwa}} and Dadra\endnote{Dadra \\ \url {https://www.taalgyan.com/taals/dadra/}} Taal patterns with eight and six beats respectively. A secondary rhythm pattern within Qawwalis is led by backing vocals clapping their hands at a periodic interval named as Taali (not to be confused with Taals). Both Tabla and Taali combine to give Qawwalis their characteristic beat structure.

Music information retrieval studies targeting dataset(s) from Indian subcontinent have been reported in much fewer numbers as compared to their western counterparts.\citep{carnatic_music} evaluated fundamental pitch of vocals as well individual instruments on six different datatsets focusing solely on classical music. Efforts to categorize Indian music into genres can also be found in internet\endnote{Indian Music Genre Dataset \\ \url{https://www.kaggle.com/datasets/winchester19/indian-music-genre-dataset}}. The open-source MusicBrainz\endnote{MusicBrainz \\ \url{https://musicbrainz.org/tag/qawwali}} includes songs tagged as Qawwali but the distinction between generalized sufi music and Qawwali in this project is unclear. None of these datasets qualify as a source of expert labeblled Qawwali songs. This paper introduces a new dataset containing Qawwali songs suited for cross-cultural music information retrieval research.

Genre recognition is a well established domain within music information retrieval reasearch. \cite{music_genre_survey} gives a good overview of datasets, classification strategies and peformance measures used within this domain. Most studies are limited to western music datasets, genre detection on eastern music datasets are few and far between. For example MIREX 2020 \citep{mirex} included a task to estimate K-Pop genre detection. However, genre detection on Indian subcontinet's music are almost non-existent. One of the earliest studies on music genre recognition is by \cite{gtzan}. They introduced the GTZAN dataset which contains 100 songs of 10 western music genres each and even with its shortcomings is still used in genre classification research. The objective of this paper is to introduce a new "Qawwali" genre that can be used alongside datasets like GTZAN. To this end, "QawwalRang" dataset presented in this paper is supported with an unsupervised genre classification algorithm  reporting results of Qawwali genre detection again the genre included in GTZAN dataset.

Rest of this paper is organized as follows: Next section presents the "QawwalRang" dataset, sources used for compiling the dataset, selection criteria and artists whose performances are included. Section 3 proposed an unsupervised classification algorithm used to label songs as Qawwali or otherwise. It details the features used to individually detecting Tabla and Taali components before making an binary genre classification decision. Section 4 presents the results of applying genre classification on QawwalRang and GTZAN datasets. Observation on which western musical genre may be more similar to Qawwali are also included in this section. Last section of Reproducible describes the software used to build the dataset as well as an implementation of classification algorithm.

\section{QawwalRang Dataset}\label{sec:data}

QawwalRang dataset is a collection of 72 expert labeled songs. Each song, typically, is an excerpt from a much longer original recording. This excerpt is re-sampled at 44.1 KHz and converted into a single (mono) channel. The resulting song is tagged with a set of metadata values capturing its origin and creation process as detailed below. 

\subsection{Selection}
Surprisingly, even though Qawwali is an old and popular genre, manual labeling of songs as Qawwali is a non-trivial task. This is partly due to close association of Qawwali with so-called "sufi" music which can be performed in other genres like Folk or Bhajan as well. Additionally, wrong labeling in online databases is also due to the improvisation aspect, artists can mix/match Qawwali with modern music forms like pop music. In order to achieve a meaningful labeling strategy, QawwalRang construction involved inducting songs with the following criteria:
\begin{itemize}
\item Song is performed by well-known/renowned Qawal artists/groups (also called Qawal parties). While collected items for this dataset, one of the aim was to stay as close as possible to the original Qawwali form, while achieving sufficient artist diversity. Including songs from artists associated primarily with Qawwali genre reduces the sample space and makes the labeling obvious both structurally and perceptually.
\item Songs must have an introductory section including both essential rhythm elements namely Tabla and Taali. These two components make Qawwali perceptually distinct as compared to other musical genres of Indian subcontinent and can be easily identified by a listening expert. In a very small minority of cases where songs start with a vocal/instrument improvisation but does contains segments with Tabla and Taali components at a later time-point, an offset is incorporated to skip to the time-segments containing Tabla and Taali components.
\item Songs performed in classical/semi-classical style have been preferred, this reduces overlap with other genres like pop, folk and film music. Songs that follow a modern mixture of Qawwali, overlapping traditional Tabla/Taali components with mainstream instruments are avoided.
\end{itemize}

After a song fulfilling the above criteria is selected, next step is to create a metadata entry for it. The metadata entry follows a key-value interface which defines the following keys:
\begin{enumerate}
\item ID: unique identifier of the song in the data-set
\item Name: song's descriptive name suitable for searching in global databases
\item Artist: performer's or Qawwal party's name
\item Location: URL of the original song from which excerpt has been taken. For songs from personal collection, samples are uploaded to an accessible place in cloud and a URL provided.
\item Start: Time-stamp in seconds within original song from where snippet was taken
\item Duration: Duration of song snippet in seconds, default is 1-minute. The default value is chosen with the consideration that some Qawwali performance can easily have a running time of 20-30 minutes.
\end{enumerate}

As a result, construction of QawwalRang dataset produces an associated metadata structure which completely specifies the dataset. Users can simply rebuilt the dataset with software operating on just this modifiable metadata structure. Please refer to the Reproducibility section for details on how this is achieved.

\subsection{Sources}
Our first preference was to source QawwalRang songs solely from personal collections. This was in part motivated by the first point in the selection criteria, namely to include as many classical Qawwali songs as possible. However poor recording quality, non-availability of old recordings and the effort of ripping material off from cassettes/compact-disks and conversion to digital format proved quite daunting. To overcome these challenges we turned to internet where a huge cache of songs is available but there also, ratio of songs fulfilling the selection criteria to the total number of Qawwali songs returned by a search engine is surprisingly low. Eventually QawwalRang songs originate from three sources i.e. internet music sharing platform(s), among others, YouTube\endnote{YouTube: \\
\url{www.youtube.com}}, author(s) personal collection and crawling the web for Qawwali songs. Table \ref{tab:sources} shows the distribution of sources in QawwalRang dataset.
\begin{table}[htpb]
\centering
  \begin{tabular}{|c | c|}
  \toprule
  \bfseries Source & \bfseries Dataset Share(\%) \\
  \hline \hline
  YouTube  & 61 \\
  \hline
  Personal & 30 \\
  \hline
  Web-Crawler & 9  \\
  \bottomrule
  \end{tabular}
  \caption{QawwalRang: Source Distribution}
\label{tab:sources}
\end{table}

\subsection{Diversity}

During compilation of presented dataset, every attempt has been made to include songs from artists with different styles, active-periods and musical preferences. Figure \ref{fig:author_dist} shows artist distribution in QawwalRang dataset. This may see skewed at first, but among the selected artists Nusrat Fateh Ali Khan (1948-1997) \citep{nusrat} and his group contribute a major share. Readers familiar with Qawwali genre will probably know that Nusrat and his group were the foremost exponents of this genre in the last quarter of previous century. They popularized Qawwali genre with tireless performances across the world throughout 70s/80s/90s and have remained widely popular and relevant even in recent times. In that sense their big share in the dataset is merited. The next big contribution comes from 'others' category which is actually 9 songs each from a different artist. Rest of the dataset is quite evenly distributed among other selected artists which brings the total number of artists in the dataset to be 23.

\begin{figure}[htbp]
  \centering
  \includegraphics[scale=1.0, width=0.95\columnwidth]{artist}
  \caption{QawwalRang: Artist Distribution}
\label{fig:author_dist}
\end{figure}

\section{Qawwali Genre Detector}\label{sec:detector}

The main motivation for developing a Qawwali classifier is to evaluate suitability of QawwalRang dataset for music information retrieval (MIR) research. We wanted to study fundamental signal properties of the Qawwali samples; to be able to draw parallels between Qawwali and western music genres. Two of the most common signal properties used in music genre detection algorithms are Constant-Q Transform (CQT) and Mel-Frequency Cepstral Coefficients (MFCC) \citep{panagakis}. Former is the time-frequency representation of an audio signal with frequency separation modeled on human ear, while the later captures the timbre quality of the signal.Instead of directly extracting these properties or features (as they are sometimes referred to) from polyphonic audio samples in the dataset, we factor in the characteristic Qawwali rhythm components namely Tabla and Taali. The idea is to first extract these components from polyphonic audio using non-negative matrix factorization, an algorithm \citep{virtanen} suitable for unsupervised sound source separation. Tabla, being the main percussion instrument and Qawwali supporting limited Taals, is expected to contribute a big chunk of spectral energy in lower octaves. This hints towards the suitability of CQT feature extracted from Tabla separated sound. Taali on the other hand is a short-time transient event without a fixed spectral distribution, while contributing a unique timbre to the Qawwali sound. Thus MFCC seems like an appropriate feature choice for Taali separated source.

The proposed Qawwali genre classification thus consists of two stages, a feature extraction and a binary classification stage. The feature extractor separates Tabla and Taali components from the raw audio data then extracts CQT and MFCC from each component respectively. Individual decisions on Tabla and Taali detection are based on heuristics for CQT pitch energy and local MFCC extrema points. This is then fed to a binary classifier which decides if the song processed by the system is classified as Qawwali or otherwise. The overall scheme is shown in Figure \ref{fig:block_dia}

\begin{figure}[htbp]
  \centering
  \includegraphics[scale=1.5, width=0.95\columnwidth]{qawali_detector}
  \caption{Qawwali genre detector.}
\label{fig:block_dia}
\end{figure}

\subsection{Feature extractor}

Features are extracted from raw audio spectrogram after separating the audio in potential Tabla/Taali source spectrograms. Let $\boldsymbol{S}$ be an ${t\times  m}$ audio spectrum matrix decomposed into sorted feature matrix $\boldsymbol{W}$ and coefficients matrix $\boldsymbol{H}$ each of dimensions ${t\times n}$ and ${n\times m}$ respectively with non-negative matrix factorization, where $n$ is the number of independent music sources the song is composed of.

\begin{align}\label{eq:eq1}
\boldsymbol{S} = \boldsymbol{W}.\boldsymbol{H}
\end{align}

We select $l \in (0,n)$ to extract Tabla spectrogram $\boldsymbol{S}_{B}$ 

\begin{align}\label{eq:eq2}
\boldsymbol{S}_{B} = \boldsymbol{W}_{B}.\boldsymbol{H}_{B}
\end{align}
where
\begin{align}\label{eq:eq3}
\boldsymbol{W}_{B} = [\boldsymbol{W}.\boldsymbol{u}_{l} ...]
\end{align}
and 
\begin{align}\label{eq:eq4}
\boldsymbol{H}_{B} = [\boldsymbol{u}_{l}^T.\boldsymbol{H} ...]
\end{align}
with $u_{l}$ being a unit column vector from Identity matrix $\boldsymbol{I}$ with size ${t\times t}$. In other words spectrogram for Tabla source is computed
by selecting columns from feature matrix $\boldsymbol{H}$ and rows from coefficient matrix $\boldsymbol{W}$. Lower column vectors are chosen for Tabla while choosing higher column vectors for Taali spectrogram $\boldsymbol{S}_{T}$ expressed as:
\begin{align}\label{eq:eq5}
\boldsymbol{S}_{T} = \boldsymbol{W}_{T}.\boldsymbol{H}_{T}
\end{align}

For Tabla detection, extracted feature is CQT energy, obtained by mapping CQT from Tabla spectrogram and taking norm/energy along second dimension. This transforms the Tabla spectrogram matrix into a feature vector represented by:
\begin{align}\label{eq:eq6}
\boldsymbol{f}_{CQT} = \lvert \lvert f\colon \boldsymbol{S}_{B}\to CQT \rvert \rvert
\end{align}

For Taali detection, a median MFCC vector is computed by obtaining normalized MFCC vector from Taali spectrogram as follows:
\begin{align}\label{eq:eq7}
\boldsymbol{f}_{MFCC} = med(\frac{f\colon \boldsymbol{S}_{T}\to MFCC}{\lvert \lvert {f\colon \boldsymbol{S}_{T}\to MFCC} \rvert \rvert})
\end{align}

\subsection{Binary Classifier}

Qawwali genre detector is a binary classifier labeling each song as Qawwali or  non-Qawwali, i.e. any other genre. It operates on the individual decisions produces by Tabla and Taali detectors, which in turn are based on extracted CQT and MFCC features. Below is a summary of individual component detection:
\begin{itemize}
	\item Tabla detector (\textbf{TD}): Does $\boldsymbol{f}_{CQT}$ represent a Tabla source? Result is a non-binary decision with three possibilities \textit{Yes(Y)/No(N)/Potential(P)}. The rationale for allowing an indeterminate decision at component level is that Tabla can be tuned to a different tonic, resulting in shift of CQT energy.
	\item TaaLi detector (\textbf{TL}): Does $\boldsymbol{f}_{MFCC}$ represents a Taali source? This is based on extrema points within MFCC vector. Result is a binary \textit{Yes(Y\textsubscript{l})/No(N\textsubscript{l)}} decision.
\end{itemize}
Overall binary classification scheme is shown in the decision-tree diagram below, where leaf nodes \textbf{Q} indicate a decision in favor of Qawwali and \textbf{NQ} is any other Non-Qawwali genre. Essentially it means the binary classifier categorizes a song as Qawwali only in two cases: first one where both Tabla and Taali were detected and an additional case where Taali has been detected form MFCC but Tabla detection from CQT was a "potential" call. 

\begin{forest}
[TD,
	[Y,
		[TL:Y\textsubscript{l}
			[\textbf{Q}]]
		[TL:N\textsubscript{l}
			[\textbf{NQ}]]
	]
	[N,
		[TL:Y\textsubscript{l}
			[\textbf{NQ}]]
		[TL:N\textsubscript{l}
			[\textbf{NQ}]]
	]
	[P,
		[TL:Y\textsubscript{l}
			[\textbf{Q}]]
		[TL:N\textsubscript{l}
			[\textbf{NQ}]]
	]
]
\end{forest}

Tabla detection attempts to fit a Gaussian distribution onto energy in CQT bins as shown below \ref{eq:eq8}

\begin{align}\label{eq:eq8}
\boldsymbol{f}_{CQT} \sim \mathcal{N}(\mu, \sigma^{2})
\end{align}

where $\mu \in [p_{1}, p_{2}]$ is the mean pitch within an octave band and $\sigma^{2} < T_{h}$ is the variance of pitch (measured in musical notes). This is akin to finding a peak in CQT energy profile, albeit giving weight to neighboring bins as well. $p_{1}$, $p_{2}$ and fixed parameters chosen at the star of the classification process where as threshold $T_{h}$ is a tunable parameter indicating within how many notes of the peak majority of energy is concentrated. Tabla is positively detected if a Gaussian curve successfully fits with $\mu \in [p_{1}, e_{n}]$ with a reasonably small threshold value. A potential decision for Tabla detection is made in case of $\mu \in [e_{n}, p_{2}]$, where a larger value of threshold is tolerated with $e_{n}$ being edge-pitch parameter tunable during experiments.  Failure to fit a distribution in either of these two ranges results in negative Tabla decision. 

For Taali detection, fitting a distribution is not plausible due to much lower number of coefficients as compared to CQT bins. Here, we attempt to find sudden change in the MFCC vector considering short bursts of hand clapping. Given $M$ element MFCC vector with MFCC-index $i \in (0, M-1)$, Taali detector searches a local extrema point i.e. either $\boldsymbol{f}_{MFCC}(i) \leq \boldsymbol{f}_{MFCC}(j)$ or $\boldsymbol{f}_{MFCC}(i) \geq \boldsymbol{f}_{MFCC}(j)$ where $j$ is a tunable parameter. Taali decision is positive in case local extrema is found at a certain $j$ index.

\section{Results \& Discussion}\label{sec:result}
Experimental results for Qawwali classifier have been obtained with two sets of parameters: fixed and tunable. Among fixed parameters is the numbers of independent sources assumed to constitute a Qawwali, expressed by ${n}$ in (\ref{eq:eq1}). It is assumed that each performance in QawwalRang dataset contains 4 independent sources namely Vocals, Harmonium, Tabla and Taali, out of which the last two are of interest to the proposed Qawwali classifier. Tabla source in (\ref{eq:eq2}) is constructed with first and second column vectors of feature matrix while second and fourth column vectors constitute Taali source in (\ref{eq:eq5}). Second fixed parameter of the experiments is the number of CQT bins in (\ref{eq:eq6}) which is fixed at 84 ranging between musical notes (midi-numbers) C1(24) and C8(108). Tabla detection is based on CQT measurements between third and fourth octaves which means $p_{1}$ and $p_{2}$ in (\ref{eq:eq8}) are set to C3(50) and G4(67) respectively. The last fixed parameter is the number of MFCC used in Taali detection meaning $\boldsymbol{f}_{MFCC}$ in (\ref{eq:eq7}) is a one dimensional vector with $M$=13 coefficients. Source separation, CQT and MFCC are computed with corresponding functions from librosa \citep{brian_mcfee_2022_6097378}, while curve fitting on Tabla CQT to find a peak of CQT energy in third and fourth octave has been done with the help of lmfit optimization library ~\citep{newville_matthew_2014}.

The tunable parameters of the experiments (with reference to previous section) are: edge-note within third octave ($e_{n}$), CQT energy threshold in third ($T_{3h}$) and fourth octave ($T_{4h}$) for Tabla detection and index ($j$) of local MFCC extrema for Taali detection. The proposed Qawwali classifier is governed by these four design parameters. In this section we discuss the impact of each of these parameters in detail leading to a set of values used for classifying Qawwali against other genres in GTZAN dataset. An assessment on each of these parameters in studied in light of four figures of merit (FOM) \citep{music_genre_survey}: \textit{Accuracy}, \textit{Recall}, \textit{F-measure} and \textit{Precision}.

The main objective of these experiments is to find suitable values for tunable parameters such that proposed Qawwali classifier is able to separate songs in QawwalRang dataset from the western music samples in GTZAN dataset; properly tuned parameters should also produce results offering insights into similarities between western music genre(s) and Qawwali. Figure \ref{fig:src_edge} shows the FOM plotted against various values of edge-notes within third octave. In order to drive these results, other variable parameters are given some default values, energy threshold in third and fourth octaves is set to 3 and 1 respectively while local extrema point for Taali detection is assumed to be the middle.  We see that all four metrics saturate beyond midi-number 53 (F3) which effectively indicates an interval of [C3,F3] within third octave. Recalling the binary classifier operation from previous section, this means that Qawwali detector should attempt to find a peak in CQT energy within this interval for positively detecting Tabla.

Figure \ref{fig:src_o3} shows FOM plot against a threshold of CQT energy spread within third octave. Here, we observe that a threshold of energy spread higher than 4 does not improve \textit{Recall} (positive Qawwali detection) but reduces \textit{Precision} and \textit{Accuracy}, this is obvious because choosing a larger energy spread as threshold increased False Positives by pushing more songs from other genres to be detected as Qawwalis due to wrongly detecting Tabla presence.
\begin{figure}[htbp]
  \centering
  \includegraphics[scale=1.0, width=0.95\columnwidth]{edge}
  \caption{Tabla Detection: Edge note in third octave}
\label{fig:src_edge}
\end{figure}
\begin{figure}[htbp]
  \centering
  \includegraphics[scale=0.75, width=0.95\columnwidth]{o3}
  \caption{Tabla Detection: CQT variance in third octave}
\label{fig:src_o3}
\end{figure}
\begin{figure}[htbp]
  \centering
  \includegraphics[scale=0.75, width=0.95\columnwidth]{o4}
  \caption{Tabla Detection: CQT variance in fourth octave}
\label{fig:src_o4}
\end{figure}
Last two plots suggest that a good bet for Qawwali detector is to decide in the favor of Tabla if it is able to fit a Gaussian distribution between [C3, F3] with variance under 4. However as explained in the section, Tabla CQT energy concentration can vary anywhere between third and fourth octave; a "Potential" decision is made in case an energy peak is detected within fourth octave. Figure \ref{fig:src_o4} shows the impact of choosing a CQT energy threshold for Tabla detection in fourth octave. We observe that selecting a higher threshold results in very good recall as almost all Qawwalis in the dataset exhibit Tabla CQT energy within third or fourth octave. At the same time, the classifier identifies a lot of False Positive with a large threshold. These experiments suggest a suitable value of 14 achieving a neutral F1-score.

The last tunable parameter is a local extrema within MFCC vector for Taali detection. That is to say that Qawwali detector decides in the favor of Taali if the local maximum/minimum value falls on a specific coefficient index ($j$). Figure \ref{fig:src_mfcc} shows the impact on choosing local extremum point to be for $5^{th}$, $6^{th}$ and $7^{th}$ coefficient. Choosing another point resulted in vastly inferior performance results. Looking at the FOM metrics from Figure \ref{fig:src_mfcc}, $6^{th}$ coefficient was chosen to identify Taali source within the song.
\begin{figure}[htbp]
  \centering
  \includegraphics[scale=0.75, width=0.95\columnwidth]{taali_mfcc}
  \caption{Taali Detection: MFCC local extremum impact}
\label{fig:src_mfcc}
\end{figure}

Based on these results, tunable parameters for Qawwali detector were determined to be: $e_{n}=3$, $T_{3h}=4$, $T_{4h}=14$ and $j=4$. Figure \ref{fig:src_genre} shows the performance of proposed Qawwali classifier using these parameters against music genres of GTZAN dataset.First thing to note is that a great majority of songs are correctly \textbf{NOT} classified as Qawwali. The mean \textit{Accuracy} rate across GTZAN genres comes out to be 90.6\% evaluated across 10 genres. Also a reasonable number of songs in QawwalRang dataset are correctly identified as Qawwali with the \textit{Recall} standing at 76\%. Really interesting are the results for 'blues' and 'classical' genre(s) showing a high false positive rate of 24\% and 29\% respectively. This suggests that in terms of fundamental music properties of CQT and MFCC, songs in these genre have a high likelihood to be identified as Qawwali. A closer look at the results from these genres indicate that increase in False Positive cases is due to songs in these categories returning a positive Taali detection with a suspicion of Tabla. This indicates that atleast in timbre qualities songs from these genres would be perceived much closer to Qawwali as compared to other western genres. For 'classical' genre this can be explained by the fact that GTZAN dataset has many songs as acoustic piano instrumentals. This has the potential of overlapping with harmonium, the leading melody instrument in Qawwali performances. For 'blues' genre, characteristic blues rhythm sounding similar to Taali structure in Qawwali is a possible explanation. 

\begin{figure}[htbp]
  \centering
  \includegraphics[scale=1.0, width=0.95\columnwidth]{genreA}
  \caption{Qawwali detection per Genre}
\label{fig:src_genre}
\end{figure}

Perhaps a noteworthy point about these results; they are not about a generalized genre detector being able to recognize a new Qawwali genre. Instead they highlight that QawwalRang songs possess fundamental musical properties distinguishing these songs apart from western music genres. In that sense the proposed dataset can be employed by state of the art genre detection studies aiming to incorporate songs from non-western musical sources. What these results also highlight is that with more intelligent feature extraction and/or classification techniques, there is a huge scope of obtaining detailed insights into cross-cultural music information retrieval, as long as appropriately labeled datasets are used. 

\section{Reproducibility}
Accompanying the dataset are command-line interface (CLI) programs that can used to reproduce the results presented in this study. The software includes a python program to construct, extend or filter songs in the dataset. It operates on a JSON \citep{JayashreeBhat}{json} formatted metadata file, editable by users before building a dataset. As explained in Section 2, the metadata interface allows user to specify an offline songs directory/folder. This feature can be used to rapidly build a dataset from songs locally available songs. Song entries in metadata structure specifying online mode are downloaded, processed and then included in the resulting dataset. A python implementation of proposed Qawwali detector is also part of the released software. It operates on the dataset constructed by the counterpart data builder program. The Qawwali detector program produces classification results per song with a summary at the end showing how many songs within the dataset were classified as Qawwali. This program can optionally be run in a feature-extraction mode, during which features are extracted and saved locally. This mode skips classification. This can be useful later with the program running classifier on multiple feature-sets each of which may belong to a different genre. The Qawwali detector program also supports a comparison option in which Qawwali classification results from various genres can be compared against each other. TODO: proper citation required DOI: \url{https://github.com/fsheikh/QawwalRang} 
\section{Conclusion \& Future Work}
Summarize and itemize future work 
%%%%%%%%%%%%%%%%%%%%%%%%%%%%%%%%%%%%%%%%%%%%%%%%%%%%%%%%%%%%%%%%%%%%%%%%%%%%%%%%
% Please do not touch.
% Print Endnotes
\IfFileExists{\jobname.ent}{
   \theendnotes
}{
   %no endnotes
}
%%%%%%%%%%%%%%%%%%%%%%%%%%%%%%%%%%%%%%%%%%%%%%%%%%%%%%%%%%%%%%%%%%%%%%%%%%%%%%%%

\section*{Acknowledgements}

Any acknowledgements must be headed and in a separate paragraph,
placed after the main text but before the reference list.

%%%%%%%%%%%%%%%%%%%%%%%%%%%%%%%%%%%%%%%%%%%%%%%%%%%%%%%%%%%%%%%%%%%%%%%%%%%%%%%%
% Bibliography
%%%%%%%%%%%%%%%%%%%%%%%%%%%%%%%%%%%%%%%%%%%%%%%%%%%%%%%%%%%%%%%%%%%%%%%%%%%%%%%%

% For bibtex users:
\bibliography{TISMIRtemplate}

% For non bibtex users:
%\begin{thebibliography}{citations}
%
%\bibitem {Author:00}
%E. Author.
%``The Title of the Conference Paper,''
%{\it Proceedings of the International Symposium
%on Music Information Retrieval}, pp.~000--111, 2000.
%
%\bibitem{Someone:10}
%A. Someone, B. Someone, and C. Someone.
%``The Title of the Journal Paper,''
%{\it Journal of New Music Research},
%Vol.~A, No.~B, pp.~111--222, 2010.
%
%\bibitem{Someone:04} X. Someone and Y. Someone. {\it Title of the Book},
%    Editorial Acme, Porto, 2012.
%
%\end{thebibliography}

\end{document}
